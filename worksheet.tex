\documentclass{dcbl/challenge}

\setdoctitle{Libraries \& Linking}
\setdocauthor{Stephan Bökelmann}
\setdocemail{sboekelmann@ep1.rub.de}
\setdocinstitute{AG Physik der Hadronen und Kerne}
\usepackage{listings}
\usepackage{hyperref}
\lstset{frame=tb,
    language=C,
    aboveskip=3mm,
    belowskip=3mm,
    showstringspaces=false,
    columns=flexible,
    basicstyle={\small\ttfamily},
    numbers=none,
    numberstyle=\tiny\color{gray},
    keywordstyle=\color{blue},
    commentstyle=\color{dkgreen},
    stringstyle=\color{red},
    breaklines=true,
    breakatwhitespace=true,
    tabsize=3
}

\begin{document}

\section*{Exercises}
\begin{aufgabe}
    Write a program called \texttt{main1.c}, that calls an arbitrary pure-function with the prototype \texttt{int f(int x, int y)}.
    Have the main only have one line, which returns the result of the function to the shell.
    First, define the function before the main, and run \texttt{gcc -E main1.c -o 1.i}.
    This will only execute the preprocessor.
    Then, copy only the function \texttt{f(int, int)} with its full definition into a new file called \texttt{f.c}, that is located in the same directory as \texttt{main.c}.
    Write a second \texttt{main2.c} that uses the \texttt{\#include "f.c"} preprocessor-directive to include the file \texttt{f.c} and run the program \texttt{gcc -E main2.c -o 2.i}.
    Use \texttt{vimdiff} to inspect both of the files.
    Explain what you see.
\end{aufgabe}

\begin{aufgabe}
    Run \texttt{gcc -c main1.c -o 1.o}, and inspect the object file using \texttt{objdump -D -t -s -x 1.o}.
    Inspect the section \texttt{.text} of the object file starting with the address of your main function.
    Can you find the function call? 
    What does it say?
    What would you expect the object-file of \texttt{main2.c} to look like?
\end{aufgabe}
    
\begin{aufgabe}
    Write a file called \texttt{f.h}, that only contains the following function declaration statement: \texttt{int f(int,int);};
    Write a file called \texttt{f.c}, that contains the following function definition: \texttt{int f(int x, int y)\{return x+y;\}}.
    Write a file called \texttt{main3.c}, that contains the following main function definition:
    \begin{lstlisting}
    #include "f.h" 
    
    int main(){
        return f(1,2);
    }
    \end{lstlisting}
    Run \texttt{gcc -c main3.c -o 3.o \&\& gcc -c f.c -o f.o}
    Inspect both files with \texttt{objdump}.
    What happend to the jump, that you found in the object-file of \texttt{1.o}?
\end{aufgabe}

\begin{aufgabe}
    Try to compile \texttt{gcc main3.c -o 3.elf}. Why does the linker crash?
\end{aufgabe}

\begin{aufgabe}
    Try to compile it again but additionally, hand over the object-file \texttt{f.o} as in the following command: \texttt{gcc main3.c f.o -o 3.elf}.
    Explain why the linker doesn't crash now.
\end{aufgabe}

\begin{aufgabe}
    You should now have figured out, that the linker serves two purposes:
    \begin{enumerate}
        \item It links functions defined in the object-files
        \item It creates an executable with appropriate start-up code from the operating-system.
    \end{enumerate}
    Write a file called \texttt{main4.c}, that contains the following main function definition:
    \begin{lstlisting}
    #include <math.h>
    int main(){
        return sqrt(4);
    }
    \end{lstlisting}
    Try compiling it with the two following commands:
    \begin{enumerate}
        \item \texttt{gcc main4.c -o 4.elf}
        \item \texttt{gcc main4.c -o 4.elf -lmath}
    \end{enumerate}
    What do you see and why does this happen?
\end{aufgabe}

\begin{aufgabe}
    What is the difference between the following:
    \begin{enumerate}
        \item \texttt{\#include "library.h"}
        \item \texttt{\#include <library.h>}
    \end{enumerate}
    How does this relate to calling the linker?
\end{aufgabe}

\begin{aufgabe}
    Create a new repository on GitHub, called \texttt{personal\_helpers}.
    Clone the empty repository to your home-directory.
    Navigate into it and create a new file called \texttt{personal\_helpers.h} as well as a new file called \texttt{personal\_helpers.c}.
    Declare the following function in \texttt{personal\_helpers.h}:
    \begin{lstlisting}
    int f(int, int);
    \end{lstlisting}
    Define the function in \texttt{personal\_helpers.c}:
    \begin{lstlisting}
    int f(int x, int y){
        return x+y;
    }
    \end{lstlisting}
    Commit this change to your repository, and push it to your GitHub account.
    Now run the following commands:
    \begin{lstlisting}
    gcc -c personal_helpers.c -o personal_helpers.o
    ar -rc libpersonal_helpers.a personal_helpers.o
    sudo cp libpersonal_helpers.a /usr/local/lib
    sudo cp personal_helpers.h /usr/local/include
    \end{lstlisting}
    With these commands, we compiled our personal helper library, wie build a linker-readable archive, and copied it to \texttt{/usr/local/lib} and to \texttt{/usr/local/include}.
    Go to your home-directory and make a new directory called \texttt{any\_new\_project}.
    Write a file called \texttt{main5.c} that contains the following main function definition:
    \begin{lstlisting}
    #include <personal_helpers.h>
    int main(){
        return f(1,2);
    }
    \end{lstlisting}
    Compile it with the following command:
    \begin{lstlisting}
    gcc main5.c -o 5.elf -l personal_helpers
    \end{lstlisting}
    Can you explain, what the \texttt{-l} option does?
\end{aufgabe}

\begin{aufgabe}
    Since the process of compiling and archiving libraries can be tedious, we'd like to automate it for us.
    Automating the build process of C-programs is usually done by a tool called \texttt{make}.
    It functions as sort of a shell-script, but with a little different syntax and is always called \texttt{MAKEFILE} or \texttt{makefile}.
    A makefile typically contains so called recipies. 
    A recipie is a set of commands, that are executed in a certain order.
    It consists of the following parts:
    \begin{itemize}
        \item The name of the target
        \item The name of the source-files it depends on
        \item The commands to be executed
    \end{itemize}
    A recipie is written as follows:
    \begin{lstlisting}
    TARGET: SOURCES
        COMMANDS
    \end{lstlisting}
    The name of the target is typically the name of the file, that is to be created, but we can also use other names.
    If we would like to use other names, that will not produce files with the same name, we mark them as \texttt{.PHONY}.
    For our example, the makefile could look like this:
    \begin{lstlisting}
    .PHONY: build install clean
    
    # This compiles the source-files
    build: personal_helpers.c personal_helpers.h
        gcc -c personal_helpers.c -o personal_helpers.o
        ar -rc libpersonal_helpers.a personal_helpers.o
    
    # This installes the library and must be run with sudo-permissions
    install: build
        cp libpersonal_helpers.a /usr/local/lib
        cp personal_helpers.h /usr/local/include
    
    # This removes the object-files
    clean:
        rm personal_helpers.o
        rm libpersonal_helpers.a
    \end{lstlisting}
    Write this into a file called \texttt{makefile} and make sure, that indented line-endings are done with tabs, not with spaces.
    Run \texttt{make build}, \texttt{sudo make install} and then \texttt{make clean}.
    Add and commit your makefile to your repository and push it.
    This library shall now contain all of the helper-functions you will write over the next semester.
    This makefile is merely an example. 
    There are many more things, that you can automate with makefiles.
\end{aufgabe}


\section*{Annotations}
\begin{enumerate}
    \item Full blown MAKEFILE for this exercise: \href{https://gist.github.com/bjoekeldude/738a19d7309a6cad95d4ae75f1b702a8}{Gist}
\end{enumerate}

\end{document}
